{\large \textbf{Paper 1: Outline }} 
\section{Introduction}
\begin{enumerate}
\item Motivation : Decoherence a challenge for large scale quantum computing. Spatial noise fields correlate qubit operations on hardware. 
\item Mapping spatial noise is a first step in developing optimised controls and error suppressing protocols. Using sensing qubits embedded in architecture is an efficient approach for characterising low level quantum computing hardware. 
\item Inspired by classical SLAM, we use sensing qubits within an architecture to optimally reconstruct maps of noise/error in space, using a new algorithmic protocol, `qslam'. 
\item qslam is a non linear particle filter modified to share state information between particles. Operates by updating local state information before sharing information globally to  reconstruct spatial  map. Sharing mechanism uses Born's rule which links data to values of the map at any location.
\item We find qslam does spatial field reconstruction with less measurements than brute force. Confirmed in numerical testing using simulated and real data.  
\end{enumerate} 
\section{qslam framework}
\begin{enumerate}
	\item Physical setting:  a spatial arrangement of qubits in 2D subject to unknown classical field (dephasing) 
	\item Our objective is to optimally reconstruct dephasing field using single qubit measurements.
	\item Two algorithms: Brute force approach measures each qubit, and qslam chooses where to measure next
	\item Operating principle: qslam is an iterative maximum likelihood optimisation.For each msmt, we update state variables locally, and then share locally optimal state information in a neighbourhood  Depicted as a schematic in the FIG (below) 
	\item FIG: Define map variable, define neighbourhood, show how msmt information updates state variable, mediated by a controller. 
	\item Describe controller: qslam decides what spatial region requires more measurement data 
	\item Numerics: qslam implemented as modified particle filtering with 2 types of particles - one type carries state information, and the other type discovers optimal neighborhood size for sharing information at any given qubit location
	\item Explain expected qslam behaviour for simple examples e.g. uniform fields, spatially varying fields. 
	\item Non-trivial sharing: qslam sharing mechanism is tunable. When sharing  is turned off via hyper parameters $\lambda_0, \lambda_1 = 0 $, qslam reduces to  brute force  
\end{enumerate} 

\section{Results}

\begin{enumerate}
	\item Simulations approach:  compare spatial dephasing maps between a true dephasing field and output map from an algorithm (qslam, brute force), calculate expected error
	\item Error metrics: RMS error scores absolute differences between maps, SSIM scores structural similarity between maps
	\item FIG: Linear array of qubits in low/high field 
	\item FIG: 2D array of qubits in true field of 4 regions. Show control path when neighbourhoods are 2D
	\item FIG: 2D array of qubits in asymmetric Gaussian smoothly varying true field with control path. Show control path and show optimised $\lambda_0, \lambda_1 \neq 0$.
	\item FIG: Simulations with experimental data on a string of trapped ions. Show that qslam works on real experimental data for Ramsey-like measurements.  Calculate B-field gradient from qslam state variables and compare with expt. measurements.
\end{enumerate} 

\section{Discussion}

\begin{enumerate}
	\item Key finding 1: For the same error level, qslam acheives map reconstruction better than brute force. qslam is unbiased estimator.
	\item Key finding 2: Information sharing non trivial as optimal $\lambda_0, \lambda_1 \neq 0$
	\item Iterative ML procedure and sharing mechanism are enabled by Born's rule that links measurement data directly to the local value of the map. 
\end{enumerate} 

\section{Conclusion}

\begin{enumerate}
	\item Optimal spatial map reconstruction with significant resource savings over brute force approaches, using novel particle filtering approach with information sharing. \item Enables in-situ noise characterisation in quantum hardware by making use of idle qubit resources
	\item  Extension: appropriate design of transition probability distributions and adaptive $\lambda_0, \lambda_1$ enables tracking over `locally stationary' time varying fields. 
\end{enumerate} 

\clearpage
{\large \textbf{Paper 2: Outline }}

\begin{enumerate}
	\item Introduction (TBC: Lit review)
	\item Physical Setting
	\item qslam Framework
	\item Iterative Maximum Likelihood Algorithm
	\item Results (TBC)
	\item Discussion (TBC)
	\item Conclusion (TBC)
	\item Appendix A: Theoretical Background
	\item Appendix B: Pseudo-Code (TBC: update notation)
\end{enumerate}

\section{Revised Intro}
\begin{enumerate}
	\item Decoherence acting on quantum computing hardware is a key barrier to realising at-scale, high fidelity quantum computations 
	\item In realistic operating environments, noise sources are often correlated in space or time and these noise drifts typically violate basic assumptions in quantum computation;  thereby correlating qubit operations and qubit measurements that would otherwise be statistically independent in any ideal procedure. 
	\item  Mapping noise/error in space gives a way to deploy optimized controls in an effort to suppress errors.  Using sensing qubits within an architecture is widely considered an efficient approach to low level physical-hardware characterization.
	\item The availability of free or idling single qubits on hardware while quantum computations are being carried out represents an opportunity to use these qubits for noise-characterisation measurements.
	\item The inter-weaving of noise-characterisation measurements with quantum computations as qubits are absorbed or released from quantum circuits enables us to characterised the underlying noise field affecting hardware in realistic operating environments.
	\item You want to build a spatial map of the underlying error-inducing noise in space and you want to do this with maximum efficiency in the number of measurements required.  Set up the idea of comparing against brute force measurement of all qubits.
	\item The optimal reconstruction of an arbitrary unknown environment using local sensor measurements is well considered classically in the field of autonomous robotics, but has not been considered as a control technique for quantum computing applications.
	\item  The focus of this manuscript, then, is to learn spatial dephasing noise correlations from binary single qubit measurements by choosing which qubit to projectively measure next on a 2D spatial grid. We resolve this inference problem inspired by the existence of similar problems in classical robotic control.
	\item 1) You present a SLAM inspired algorithm for mapping in multi-d spaces, called QSLAM
	2) How is it constructed?
	3) It leverages born rule
	4) Convergence proofs provided
	5) Numerically implemented
	6) What do you observe
\end{enumerate}

\iffalse
\section{Introduction}

\begin{enumerate}
	\item Decoherence acting on quantum computing hardware is a key barrier to realising at-scale, high fidelity quantum computations 
	\item In realistic operating environments, noise sources are often correlated in space or time and these noise drifts typically violate basic assumptions in quantum computation;  thereby correlating qubit operations and qubit measurements that would otherwise be statistically independent in any ideal procedure. 
	\item The availability of free or idling single qubits on hardware while quantum computations are being carried out represents an opportunity to use these qubits for noise-characterisation measurements.
	\item The inter-weaving of noise-characterisation measurements with quantum computations as qubits are absorbed or released from quantum circuits enables us to characterised the underlying noise field affecting hardware in realistic operating environments.
	\item The optimal reconstruction of an arbitrary unknown environment using local sensor measurements is well considered classically in the field of autonomous robotics, but has not been considered as a control technique for quantum computing applications.
	\item The focus of this manuscript, then, is to learn spatial dephasing noise correlations from binary single qubit measurements by choosing which qubit to projectively measure next on a 2D spatial grid. We resolve this inference problem by borrowing classical solutions to the simultaneous localisation and mapping (SLAM) problem in the field of autonomous robotics and sensing. 
	\item While these analogies pertaining to a `map' and `local sensors' are useful as a high level overview of the key concepts in the inference problem at hand, we depart from the classical SLAM case in several non-trivial ways. 
	\item Of these, the key challenge is that there is no quantum mechanical equivalent of classically scanning the environment without forcing the qubits to collapse into a particular state. 
	\item Secondly, in classical SLAM,  data association means that one unambiguously  extracts `map features' from a physical scan measurements, where map features are sharply discontinuous from the background (e.g. local features, such as a table, or edge features, such as a wall). In contrast, we assumed physical sources cannot produce a discontinuous features in dephasing field - namely, our environmental map is  continuously varying everywhere.
	\item For the simple case of qubits under dephasing, we exploit the existence of a theoretical data association function - the quantum mechanical Born rule - to adapt classical simultaneous mapping and localisation techniques for classical control applications for quantum computing. The resulting framework, denoted `qslam', is numerically implemented via a particle filter and tested for correctness using simulated data for a range of realistic operating environments. We further link `qslam' to convergence results for discrete time stochastic non-linear filtering in existing classical literature.  
	\item Structure of document as follows
\end{enumerate}
\fi
\input{./detailed_outline.tex}

\section{Results}

\begin{enumerate}
	\item High level simulation approach; including `naive' 
	\item Performance metrics: RMS Error and SSIM score for comparing maps. Why SSIM?
	\item FIG: Linear array of qubits in step function true field 2 regions. Show a simple case where we find out that the field is different in 2 regions over hardware; qslam does this in less measurements than naive:
	\begin{enumerate}
		\item true error (SSIM) score vs. number of msmts. Legend: qslam, naive. Inset: percentage of msmts qslam took to get the same error level as naive
		\item true error (SSIM) score for qslam vs. extent of sharing [0,1] where 0:= don't trust physical msmts and 1:= only trust physical msmts. $\lambda_0, \lambda_1 \neq 1$ is evidence for non-trivial information sharing
		\item example images of maps with qslam v. naive
	\end{enumerate}
	\item FIG: 2D array of qubits in step function true field with 4 regions. Show that when neighbourhoods are 2D, the controller is non-trivially selecting which qubit to measure next:
	\begin{enumerate}
		\item true error (SSIM) score vs. number of msmts. Legend: qslam, naive. Inset: percentage of msmts qslam took to get the same error level as naive
		\item example of a control path (i) with all qubits available and (ii) some qubits unavailable. show resulting error rates and example images for (i) and (ii).
		\item controller variance vs number of measurements - controller variance should decrease as a function of number of measurements ??
		\item show case (i) is not just stacking of a linear array since spatial sharing is 2D	
	\end{enumerate}
	\item FIG: 2D array of qubits in asymmetric Gaussian smoothly varying true field. Since this is the only figure where continuously varying assumption is not violated,  the data should provide the the main numerical evidence for correctness of qslam
	\begin{enumerate}
		\item true error (SSIM) score vs. number of msmts. Legend: qslam, naive. Inset: percentage of msmts qslam took to get the same error level as naive
		\item true error (SSIM) score for qslam vs. extent of sharing [0,1] where 0:= don't trust physical msmts and 1:= only trust physical msmts.
		\item for two qubit locations, deep dive into posterior distribution of $F_t$, $R_t$ and true errors and explain the shape. If posterior distribution of $F_t$, $R_t$ in simulations can be explained by theory, then qslam correctness is likely
	\end{enumerate}
	\item FIG: Simulations with experimental data on a string of trapped ions. Show that qslam works on real experimental data for Ramsey-like measurements.  
	\begin{enumerate}
		\item  qslam estimated error vs. naive estimated error as a function of the number of msmts
		\item Calculate B-field gradient from qslam state variables and see if it matches Cornelius's value.
	\end{enumerate}
\end{enumerate}

\section{Discussion}
\section{Conclusion}
