Decoherence remains a key challenge in the development of large scale quantum computing architectures. Contemporary physical hardware for quantum computing often involves multi-qubit arrays embedded in chip- or wafer-like architectures analogizing classical processors in industry \cite{yao2012scalable,monroe2014large,veldhorst2017silicon,jones2012layered,kielpinski2002architecture,franke2019rent}. Characterisation and control of individual qubits embedded in hardware is impeded by the presence of unknown stray fields (noise) that manifest by correlating single qubit operations or measurements that should otherwise remain independent or by eroding coherent interactions by randomisation. In addition to single qubit control, other procedures such as scheduling and compilation protocols are increasingly critical components for overall hardware performance \cite{venturelli2018compiling,murali2019noise,shi2019optimized,venturelli2018optimization,tannu2018case}. Mapping spatial noise within these computing architectures is a first step in developing optimised controls and error suppressing protocols for multi-qubit physical settings. A particularly efficient approach is to exploit the dynamical availability of idling or free qubit resources by scheduling in-situ noise characterisation measurements for low level quantum computing hardware. 

Inspired by classical simultaneous mapping and localisation (SLAM) protocols \cite{cadena2016past,bergman1999recursive,stachniss2014particle,durrant2006simultaneous,bailey2006simultaneous,murphy2000bayesian,howard2006multi,thrun2005probabilistic,thrun1998probabilistic}, we borrow ideas from autonomous learning to optimally reconstruct a map of a classical spatial noise field in real time. This framework, denoted Quantum SLAM (QSLAM), seeks to build a map of unknown spatial fields using fewer sensing measurements that the naive approach of scheduling sensing measurements globally for all qubits on hardware. QSLAM  discovers small subsets of qubits on the hardware to find an optimal “spreading” of information in space, such that a map can be built over the array without having to measure all qubits. These subsets of qubits are grouped by discovering optimal `neighborhoods' during state estimation, and the resultant information sharing mechanism enables QSLAM to learn the unknown field faster than naively and repetitively measuring qubits everywhere. Unlike a typical classical SLAM problem, QSLAM exploits its access to a quantum mechanical Born rule to link incoming data streams with local updates to the map and facilitates information sharing in regions on the map. The existence of an apriori theoretical rule for associating data to the map enables QSLAM to find the iterative maximum likelihood solution over the space of all possible maps and all possible neighborhoods in the space of continuous random variables - a space that is too large for an inference to be tractable otherwise. 

We implement the QSLAM framework via a novel two-layer particle filter, followed by an autonomous real-time controller. Collectively, the algorithm iteratively builds a map of the noise field  in real time by maximising the information utility obtained from each physical measurement. The particle filter (cf. standard \cite{doucet2001introduction}, or SLAM implementations \cite{beevers2007fixed,grisettiyz2005improving,poterjoy2016localized})  contains two layers - the first layer for optimally updating the map locally at the measured qubit, and a second layer, for discovering an optimal region for sharing information with neighboring qubits. The optimal output from the particle filter is used by the  controller to choose the location for the next physical measurement, namely, in a region where map reconstruction is maximally uncertain. For engineered noise fields with simulated data, our results show that the brute force measurement approach uses between $2$x - $25$x more measurements than QSLAM to reconstruct the noise map for a range of acceptable error thresholds in both 1D and 2D spatial configurations of 25 qubits subject to noise fields with different rates of spatial variation. We apply QSLAM to characterise stray magnetic field gradients over a linear array of six qubits in an ion trap using real experimental data and achieve a $4$x gain over brute force measurement in offline simulations.

In the material to follow, \cref{sec:qslam} introduces the background and key operating principles of the QSLAM framework. Results from engineered data and offine simulations using experimental data are presented in \cref{sec:results} and their interpretation as evidence for algorithmic correctness is discussed in \cref{sec:discussion}. We conclude in \cref{sec:conclusion} by highlighting our key results and by outlining extensions to time varying environments.

 

