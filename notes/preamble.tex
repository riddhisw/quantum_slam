\usepackage{amssymb,amsmath}
\usepackage{graphicx,import}
\usepackage{placeins}
\usepackage{hyperref}
\usepackage{color}
\usepackage{longtable}
\usepackage{amsthm}
\usepackage{algorithm}
\usepackage{algpseudocode} 
\usepackage[nameinlink,poorman]{cleveref}

% Make a list of notation / nomenclature
\usepackage{nomencl}
\makenomenclature
% First time - invoke makeindex via commandline: makeindex <filename>.nlo -s nomencl.ist -o <filename>.nls

% #############################################################################
%  GLOBAL PACKAGE OPTIONS AND STYLE 
% #############################################################################
% % Define paragraph spacing and line spacing
% \setlength{\parindent}{4em}
\setlength{\parskip}{20em} 
% \renewcommand{\baselinestretch}{1.5}
% #############################################################################
% newtheorem and amsmath options 
% #############################################################################
%Define format for defintions
\newtheorem{theorem}{Theorem}%[section]
% set counters to be the same as theorem
\newtheorem{defn}[theorem]{Definition}
\newtheorem{azm}[theorem]{Assumption}
\newtheorem{corollary}[theorem]{Corollary}
\newtheorem{lemma}[theorem]{Lemma}
\newtheorem{proposition}[theorem]{Proposition}
\newtheorem{remark}[theorem]{Remark}

% Define paragraph spacing and line spacing
% \setlength{\parindent}{4em}
\setlength{\parskip}{20em} 
% \renewcommand{\baselinestretch}{1.5}

% refer to properties of definition

\usepackage{enumitem}
\newlist{thmprop}{enumerate}{2}
\setlist[thmprop]{label={\normalfont(\roman*)},ref=(\roman*)}

% #############################################################################
% cleverref options
% #############################################################################

\crefname{equation}{Eq.}{Eqs.}
\crefname{figure}{Fig.}{Figs.}
\crefname{table}{Table}{Tables} 
\crefname{section}{Section}{Sections}
\crefname{chapter}{Chapter}{Chapters}
\crefname{appendix}{Appendix}{Appendices}
\crefname{algorithm}{Algorithm}{Algorithms}


\crefname{theorem}{Theorem}{Theorems}
\crefname{defn}{Definiton}{Definitions}
\crefname{azm}{Assumption}{Assumptions}
\crefname{corollary}{Corollary}{Corollaries}
\crefname{lemma}{Lemma}{Lemmas}
\crefname{thmprop}{property}{properties}
\crefname{proposition}{Proposition}{Propositions}
\crefname{remark}{Remark}{Remarks}

% #############################################################################
%  GLOBAL PAPER NOTATION
% #############################################################################


% #############################################################################
%  New notation (for this paper)
% #############################################################################

% Axiomatic probability and non linear filtering notation
\newcommand{\ex}[1]{\mathbb{E}[#1]} %mathmode only - expectation values
\newcommand{\samplespace}[1]{\Omega_{#1}}  %mathmode only - sample space Omega 
\newcommand{\family}[1]{\mathcal{F}_{#1}}  %mathmode only - family of events 
\newcommand{\measure}[2]{\mathbb{P}_{#1}{[#2]}}  %mathmode only - measure P
\newcommand{\ofieldgen}[1]{\sigma(#1)}  %mathmode only - sigma-field generated #1
\newcommand{\ofieldA}[0]{\mathcal{A}}
\newcommand{\pr}[0]{\mathrm{Pr}}
\newcommand{\erf}[0]{\mathrm{erf}}
\newcommand{\ofield}[0]{$\sigma$-field} % text mode


\iffalse
% TODO DELETE THIS
% Notation for all probability distributions
\newcommand{\prob}[2]{\mathcal{P}_{#1} ( #2 )} % true distribution
\newcommand{\probest}[2]{\hat{\mathcal{P}}_{#1} (#2)}  % sample distribution
\newcommand{\qbornp}[3]{\hat{p}^{(#1)}_{#2,#3}} % quasi msmt at #1 due to physical msmt at #2 at time #3

% %% Msmts and controls
\newcommand{\pmsmt}[2]{d^{( #1 )}_{#2}}
\newcommand{\qmsmt}[2]{\hat{d}^{( #1 )}_{#2}}
\newcommand{\cntrl}[1]{\hat{u}_{ #1}}
\newcommand{\kernel}[0]{K}

% %% True states
\newcommand{\map}[1]{f_{ #1}}
\newcommand{\rstate}[1]{r_{ #1}}
\newcommand{\pose}[1]{s_{ #1}}
\newcommand{\mval}[2]{f^{( #1 )}_{#2}}
\newcommand{\rval}[2]{r^{( #1 )}_{#2}}
\newcommand{\qset}[2]{Q^{( #1 )}_{#2}}

% %% State estimates
\newcommand{\mest}[2]{\hat{f}^{( #1 )}_{#2}}
\newcommand{\rest}[2]{\hat{r}^{( #1 )}_{#2}}
\newcommand{\poseest}[1]{\hat{s}_{ #1}}
\newcommand{\qbornm}[3]{\hat{\hat{f}}^{(#1)}_{#2,#3}}

% Map Data Association function
\newcommand{\mapfunc}[1]{\hat{m}(#1)}

% #############################################################################
% Notation from other papers
% #############################################################################

% Define Dirac and Pauli notation
\newcommand{\ket}[1]{| #1 \rangle} %mathmode only
\newcommand{\bra}[1]{\langle #1 |} %mathmode only 
\newcommand{\op}[1]{\hat{#1}} %mathmode only - operator hat
\newcommand{\p}[1]{\hat{\sigma}_{#1}} %mathmode only - pauli operators
\newcommand{\idn}[0]{\op{\mathcal{I}}} %mathmode only - identity
\newcommand{\pj}[1]{\ket{#1}\bra{#1}} %mathmode only - projection operator
\newcommand{\pjs}[2]{\bra{#1}#2\rangle} %mathmoe only - state projection


% Define commmon linear operators
\newcommand{\dd}[0]{(t)} %mathmode only - time dependence
\newcommand{\dr}[1]{\frac{d #1}{dt}} %mathmode only - time derivative 
\newcommand{\norm}[1]{||#1||} %mathmode only - norm

\fi
